\documentclass[12pt]{article}
\usepackage[utf8]{vietnam}
\usepackage{geometry}
\usepackage{graphicx}
\usepackage{listings}
\usepackage{hyperref}
\geometry{a4paper, margin=2.5cm}

\title{BÁO CÁO BÀI TẬP\\HỆ THỐNG TRUY VẤN CÓ TĂNG CƯỜNG TRI THỨC (RAG)\\ỨNG DỤNG CHO TÀI LIỆU ĐẠI HỌC QUỐC GIA HÀ NỘI}
\date{\today}

\begin{document}
\maketitle

\tableofcontents
\newpage

\section{Giới thiệu}
\subsection{Đặt vấn đề}
Bài tập  này tập trung xây dựng một hệ thống RAG cho tập tài liệu liên quan đến các đơn vị thành viên của Đại học Quốc gia Hà Nội (ĐHQGHN), nhằm hỗ trợ trả lời câu hỏi tự động dựa trên ngữ cảnh.

\subsection{Mục tiêu}
\begin{itemize}
    \item Xây dựng pipeline đầy đủ cho hệ thống RAG: thu thập, xử lý, nhúng, truy hồi và sinh câu trả lời.
    \item Ứng dụng cho tài liệu tuyển sinh, đào tạo, ngành học của ĐHQGHN.
    \item Đánh giá hiệu quả hệ thống trên tập câu hỏi chuẩn.
\end{itemize}

\section{Cấu trúc dự án}
\begin{itemize}
    \item \texttt{data/}: Dữ liệu thô, đã xử lý, và tập test gồm câu hỏi + đáp án tham chiếu.
    \item \texttt{report/}: Báo cáo dự án.
    \item \texttt{results/}: Kết quả chạy hệ thống.
    \item \texttt{scripts/}: Tập hợp script để thu thập, xử lý và vận hành hệ thống.
    \item \texttt{scripts/rag\_system/}: Thư mục chứa các thành phần chính của hệ thống RAG:
    \begin{itemize}
        \item \texttt{embedder.py}: Sinh vector nhúng từ văn bản.
        \item \texttt{retriever.py}: Truy xuất văn bản liên quan.
        \item \texttt{reader.py}: Sinh câu trả lời từ ngữ cảnh.
        \item \texttt{main.py}: Chạy pipeline tổng thể.
    \end{itemize}
\end{itemize}

\section{Thu thập và xử lý dữ liệu}
\subsection{Thu thập dữ liệu}
Dữ liệu được thu thập từ các trang web chính thức của ĐHQGHN và các trường thành viên (ví dụ: UET, ULIS, UEB, UED). Quá trình này được thực hiện bằng script trong \texttt{scripts/data\_collection}.

\subsection{Tiền xử lý dữ liệu}
Các script xử lý bao gồm:
\begin{itemize}
    \item \texttt{clean\_data.py}: Tiền xử lý tổng hợp.
    \item \texttt{clean\_csvsangtxt.py}: Chuyển CSV sang định dạng văn bản.
    \item \texttt{clean\_datacsv.py}, \texttt{clean\_datapdf.py}: Làm sạch dữ liệu từ nhiều nguồn khác nhau.
\end{itemize}

\section{Hệ thống RAG}
\subsection{Bộ nhúng (Embedder)}
Sử dụng thư viện \texttt{HuggingFace Transformers} để tạo nhúng văn bản. Dữ liệu được chuyển đổi sang vector và lưu trữ trong \textbf{Chroma VectorStore}.

\subsection{Bộ truy xuất (Retriever)}
Truy xuất top-k tài liệu phù hợp với câu hỏi người dùng. Triển khai bằng Chroma kết hợp thuật toán similarity search cosine.

\subsection{Bộ sinh câu trả lời (Reader)}
Sử dụng mô hình ngôn ngữ (LLama) để sinh câu trả lời dựa trên tài liệu đã truy xuất.

\subsection{Pipeline tổng thể}
Chạy trong \texttt{main.py} với quy trình:
\begin{enumerate}
    \item Người dùng nhập câu hỏi.
    \item Hệ thống truy xuất tài liệu liên quan.
    \item Sinh câu trả lời dựa trên ngữ cảnh.
\end{enumerate}

\section{Đánh giá hệ thống}
\subsection{Tập kiểm thử}
\begin{itemize}
    \item \texttt{questions.txt}: Danh sách câu hỏi thực tế về ĐHQGHN và các đơn vị thành viên .
    \item \texttt{reference\_answers.txt}: Đáp án tham chiếu do con người cung cấp.
\end{itemize}

\subsection{Chỉ số đánh giá}
Sử dụng các chỉ số như:
\begin{itemize}
    \item \textbf{Exact Match (EM)}.
    \item \textbf{BLEU / ROUGE}.
\end{itemize}
Kết quả được lưu trong thư mục \texttt{results/}.

\section{Kết luận}
Hệ thống RAG do nhóm thực hiện đã cho kết quả khả quan trên tập dữ liệu nội bộ. Trong tương lai, hệ thống có thể được mở rộng để tích hợp mô hình lớn hơn và phục vụ người dùng qua giao diện web hoặc chatbot.
 1 số lưu ý nhỏ: reference_answers.txt và results có thể có 1 số câu trả lời khác nhau do câu hỏi được đưa vào trong tệp question.txt chưa rõ ràng(ví dụ như các ngành, khoa chưa nói rõ của các trường nào thuộc đại học quốc gia cho nên có thể hệ thống trả lời sẽ trả lời khác), có thể cải thiện bằng cách đưa các câu hỏi rõ ràng hơn , train tiếp 

\section*{Phụ lục}
\begin{itemize}
    \item Liên kết mã nguồn: \url{https://github.com/TuanAnh11032004}
    \item Liệt kê các thư viện trong \texttt{requirements.txt}.
\end{itemize}

\end{document}
